%-----------------------------------------------------------------------
%% Preamble by organize commitee

\documentclass[12pt, a4paper]{article}
\usepackage[margin=2.5cm]{geometry}
\usepackage{xcolor}
\usepackage{fancyhdr}
\usepackage[utf8]{inputenc}
\usepackage[brazil]{babel}
\usepackage{setspace}
\usepackage{indentfirst}
\usepackage{graphicx}
\usepackage{url} % hyper links
\usepackage{enumerate}
\usepackage{amsmath,amsthm,amsfonts,amssymb,amsxtra}
\usepackage{bm}
\usepackage[alf, bibjustif, abnt-etal-list=0]{abntex2cite}
\pagestyle{fancy}
\fancyhf{}
\lhead{$62^{\textrm{a}}$ RBras e $17^{\textrm{0}}$ SEAGRO }
\rhead{24 a 28 de julho de 2017, Lavras - MG}
\cfoot{\thepage}
\renewcommand{\headrulewidth}{0.4pt}
\addtolength{\headheight}{12.0pt}

%-------------------------------------------
% Aditional latex packages
\usepackage[bottom]{footmisc}
\usepackage{booktabs}
\usepackage{threeparttable}
\usepackage{tabularx}
\newcolumntype{C}{>{\centering\arraybackslash}X}
\DeclareRobustCommand{\rchi}{{\mathpalette\irchi\relax}}
\newcommand{\irchi}[2]{\raisebox{\depth}{$#1\chi$}}
\usepackage{float}
\makeatletter
\def\@xfootnote[#1]{%
  \protected@xdef\@thefnmark{#1}%
  \@footnotemark\@footnotetext}
\makeatother

%-----------------------------------------------------------------------
%% Init the document

\begin{document}
\onehalfspacing

%-------------------------------------------
%% Title
\begin{center}
\begin{center}
\textbf{\Large{Reparametrization of COM-Poisson Regression Models with
    Applications in the Analysis of Experimental Count Data}}\\[1em]
\end{center}
\end{center}
\vspace*{0.2cm}

% -------------------------------------------
%% Authors
\begin{flushright}
  {\bf Eduardo Elias Ribeiro Junior}
  \footnote[$\dagger$]{Contato \textit{jreduardo@usp.br}}
  \footnote[1]{Departamento de Ciências Exatas (LCE) - ESALQ-USP}
  \footnote[3]{Laboratório de Estatística e Geoinformação (LEG) -
    UFPR}\\
  {\bf Walmes Marques Zeviani} \footnote[2]{
    Despartamento de Estatística (DEST) - UFPR} \footnotemark[3]\\
  {\bf Wagner Hugo Bonat} \footnotemark[2] \footnotemark[3]\\
  {\bf Clarice Garcia Borges Demétrio} \footnotemark[1]
\end{flushright}

\vspace*{0.5cm}

%-------------------------------------------
%% Body

\noindent
The COM-Poisson model generalized the Poisson model for handling over
and underdispersed counts. This article proposes a
mean-parameterization of COM-Poisson model. The results shows that the
new parameterization is appropriate and offers better results than
conventional approaches. The empirical orthogonality between precision
and regression parameters, consequently less time to computing, and the
interpretation are the main advantagens of mean-parameterization
COM-Poisson model.

\noindent{\bf Keywords}: {\it Count data, Overdispersion,
  Underdispersion, COM-Poisson model, Likelihood inference.}.\\

\end{document}
