%-----------------------------------------------------------------------
%% Preamble by organize commitee

\documentclass[12pt, a4paper]{article}
\usepackage[margin=2.5cm]{geometry}
\usepackage{xcolor}
\usepackage{fancyhdr}
\usepackage[utf8]{inputenc}
\usepackage[brazil]{babel}
\usepackage{setspace}
\usepackage{indentfirst}
\usepackage{graphicx}
\usepackage{url} % hyper links
\usepackage{enumerate}
\usepackage{amsmath,amsthm,amsfonts,amssymb,amsxtra}
\usepackage{bm}
\usepackage[alf, bibjustif, abnt-etal-list=0]{abntex2cite}
\pagestyle{fancy}
\fancyhf{}
\lhead{$62^{\textrm{a}}$ RBras e $17^{\textrm{0}}$ SEAGRO }
\rhead{24 a 28 de julho de 2017, Lavras - MG}
\cfoot{\thepage}
\renewcommand{\headrulewidth}{0.4pt}
\addtolength{\headheight}{12.0pt}

%-------------------------------------------
% Aditional latex packages
\usepackage[bottom]{footmisc}
\usepackage{booktabs}
\usepackage{threeparttable}
\usepackage{tabularx}
\newcolumntype{C}{>{\centering\arraybackslash}X}
\DeclareRobustCommand{\rchi}{{\mathpalette\irchi\relax}}
\newcommand{\irchi}[2]{\raisebox{\depth}{$#1\chi$}}
\usepackage{float}
\makeatletter
\def\@xfootnote[#1]{%
  \protected@xdef\@thefnmark{#1}%
  \@footnotemark\@footnotetext}
\makeatother

%-----------------------------------------------------------------------
%% Init the document

\begin{document}
\onehalfspacing

%-------------------------------------------
%% Title
\begin{center}
\begin{center}
\textbf{\Large{Reparametrization of COM-Poisson Regression Models with
    Applications in the Analysis of Experimental Count Data}}\\[1em]
\end{center}
\end{center}
\vspace*{0.2cm}

% -------------------------------------------
%% Authors
\begin{flushright}
  {\bf Eduardo Elias Ribeiro Junior}
  \footnote[$\dagger$]{Contato \textit{jreduardo@usp.br}}
  \footnote[1]{Departamento de Ciências Exatas (LCE) - ESALQ-USP}
  \footnote[3]{Laboratório de Estatística e Geoinformação (LEG) -
    UFPR}\\
  {\bf Walmes Marques Zeviani} \footnote[2]{
    Departamento de Estatística (DEST) - UFPR} \footnotemark[3]\\
  {\bf Wagner Hugo Bonat} \footnotemark[2] \footnotemark[3]\\
  {\bf Clarice Garcia Borges Demétrio} \footnotemark[1]
\end{flushright}

\vspace*{0.5cm}

%-------------------------------------------
%% Body

\noindent
Dados de contagem são frequentes em estudos experimentais. Para análise
desses dados, o modelo de regressão Poisson é largamente utilizado
porém, devido a sua suposição de equidispersão é inadequado para
diversas situações. Uma alternativa paramétrica para análise de
contagens não equidispersas é o modelo COM-Poisson que, com a adição de
um parâmetro, contempla os casos de sub e superdispersão. O modelo
COM-Poisson pertence à família exponencial de distribuições e tem como
casos particulares o modelo Poisson e geométrico e como caso limite o
modelo binomial. Sua principal desvantagem é que a média e variância da
distribuição não têm forma fechada, ou seja, não há um parâmetro de
média da distribuição. Nesse artigo propõe-se uma reparametrização do
modelo COM-Poisson reescrevendo a média da distribuição como função dos
parâmetros originais, a partir de uma aproximação. O emprego do modelo
reparametrizado é ilustrado com aplicações e seus resultados são
comparados com os modelos Poisson, COM-Poisson padrão e
quasi-Poisson. Os resultados mostraram que o modelo Poisson é de fato
restritivo. O modelo COM-Poisson, por sua vez, mostrou-se bastante
flexível apresentando resultados similares aos obtidos via abordagem
quasi-Poisson. Nos estudos de caso, observou-se a ortogonalidade
empírica entre os parâmetros de precisão e média do modelo COM-Poisson
reparametrizado, o que tornou o procedimento computacional para ajuste
mais rápido. Outra vantagem é que os parâmetros de média podem ser
interpretados como razão de taxas, assim como ocorre no modelo
Poisson. De forma geral, os resultados apresentados pelos modelos
COM-Poisson reparametrizados foram satisfatórios e superiores às
abordagens convencionais. Sendo assim incentiva-se seu uso na análise de
dados de contagem.
\\

\noindent{\bf Keywords}: {\it Dados de contagem, Superdispersão,
  Subdispersão, modelo COM-Poisson, Inferência baseada em
  verossimilhança.}.\\

\end{document}
